\documentclass[12pt]{article}

\usepackage[a4paper]{geometry}
\geometry{text={6.5in,9in}, top=1.2in, nohead}

\usepackage{times}

% \usepackage[math]{easyeqn}
% \usepackage{easybmat}
% \usepackage[definevectors]{easyvector}
% \usepackage{easybmat}
\usepackage{fancyvrb,color}

\newcommand{\NUMBER}[1]{\ensuremath{#1}}
\newcommand{\METHOD}[1]{\texttt{\color{blue}#1}}
\renewcommand{\verb}{\SaveVerb[aftersave=\texttt{\color{red}\UseVerb{verb}}]{verb}}

\DefineVerbatimEnvironment{code}{Verbatim}
{fontsize=\normalsize,formatcom=\color{blue},baselinestretch=1}

\DefineVerbatimEnvironment{ncode}{Verbatim}
{fontsize=\normalsize, baselinestretch=1,
 numberblanklines=false,numbersep=5pt,numbers=left}

\title{\verb|Calculator|: an Expression Evaluator Class}
\author{Enrico Bertolazzi}
\date{\normalfont\small\itshape
     Dipartimento di Ingegneria Meccanica e Strutturale\\
     Universit\`a degli Studi di Trento\\
     via Mesiano 77, I -- 38050 Trento, Italia\\
     \texttt{Enrico.Bertolazzi@ing.unitn.it}\\
     \bigskip
     Release 4.1}

\sloppy
\begin{document}
\maketitle

\begin{abstract}
    This document presents a C++ class for run-time evaluation of
    simple symbolic expressions.  This is particularly suitable to
    facilitates the input process from free format files.  The C++
    compiler must support templates, namespaces and exceptions.
\end{abstract}

\section{Including the class \texttt{Calculator$<$...$>$}}
%%
The software consists of a single header file \verb|calc.hh|. There is no 
\verb|.a| or \verb|.so| files to be linked.
The \verb|calc.hh| facilities are available by the following inclusion 
statement:
\begin{code}
  # include "calc.hh"
  using namespace calc_load ;
\end{code}
%%
\section{Instantiating an expression evaluator.}
%%
The instantiation process of an expression evaluator requires the
specification of its template type.  The statement
%%
\begin{code}
  Calculator<double> ee ;
\end{code}
%%
instantiates the object \verb|ee| which in this case is an 
expression evaluator operating on \verb|double|s.

\section{Parsing a string}
%%
The object \verb|ee| can now be used to parse simple symbolic expressions
input as strings,
%%
\begin{code}
  bool err = ee . parse("1+sin(3.5)/4") ;
\end{code}
%%
the boolean variable \verb|err| is set \textit{true} when a parsing 
error is found. The function \METHOD{get\_value()} returns the value 
resulting from the parsing process:
%%
\begin{code}
  double res = ee . get_value() ;
\end{code}
%%
When an error is produced, an error report explaining what went wrong
can be print out by the class method \METHOD{report\_error}, as
for example in the following piece of source:
%%
\begin{code}
  ee . report_error(cout) ;
\end{code}
%%
The input string to the method \METHOD{parse} can contain 
more than one expression. Individual expressions must be 
separated by the end-of-statement separator "\verb|;|".
%%
\begin{code}
  err = ee . parse("a=1+sin(3) ; b=1-sin(3) ; sqrt(a*b)") ;
\end{code}
%%
In the case of a multiple expression parsing, the class method 
\METHOD{get\_value()} returns only the last evaluated value.
In the previous example this is \verb|a*b|.
%%
Notice also that the assignement symbol "\verb|=|" makes possible the 
memorization of intermediate expression values in other variables.

\section{Operators}
%%
The expression evaluator uses a fixed number of operators, that are
listed in order of precedence as follows
%%
\begin{itemize}
    \item[\verb|\#|]
    the rest of the string is a comment (and clearly ignored);
    
    \item[\verb|+|,\verb|-|]
    binary addition and subtraction, e.g. \verb|10+2; 3-4.2|;
    
    \item[\verb|*|,\verb|/|]
    binary multiplication and division, e.g. \verb|2.3*4.9; 2/4|;
    
    \catcode`\^11\relax
    \item[\verb|^|]
    \catcode`\^=7\relax
    power, e.g. \verb|10^4| (results $10000$);
    
    \item[\verb|+|,\verb|-|]
    unary \verb|+| and \verb|-|, e.g. \verb|+120; 12+-12|;
   
    \item[\verb|(|,\verb|)|]
    parenthesis are use to change operator precedence; for example
    the expression
    \verb|12-(2-2)| evaluates to \verb|12| while \verb|12-2-2| evaluates
    to \verb|8|;
    
    \item[\verb|;|]
    expression separator;
    
    \item[\verb|=|]
    assignement operator.
\end{itemize}

\section{Symbolic Constants}
%%
Two symbolic constants are available whose value is assigned by default:
%%
\begin{center}
  \begin{tabular}{|l|l|}
    \hline
    \verb|e|  & \texttt{2.71828182845904523536} \\
    \verb|pi| & \texttt{3.14159265358979323846} \\
    \hline
  \end{tabular}
\end{center}
%%
They can be used in symbolic expressions like the following one:
%%
\begin{code}
  e + sin(pi*0.5) ;
\end{code}
%%

\section{Predefined Functions}
%%
In the previous section we used the function \verb|sin|. There are a number of
predefined functions which can be used in symbolic expressions. The following
table lists them.
%%
\begin{center}
  \begin{tabular}{|l|l|}
    \hline
    \verb|abs(x)|     &   absolute value of \verb|x| \\
    \verb|pos(x)|     &   positive part of \verb|x| \\
    \verb|neg(x)|     &   negative part of \verb|x| \\
    \verb|cos(x)|     &   cosine of \verb|x| \\
    \verb|sin(x)|     &   sine of \verb|x| \\
    \verb|tan(x)|     &   tangent of \verb|x| \\
    \verb|asin(x)|    &   arcsin of \verb|x| \\
    \verb|acos(x)|    &   arccos of \verb|x| \\
    \verb|atan(x)|    &   arctan of \verb|x| \\
    \verb|cosh(x)|    &   hyperbolic cosine of \verb|x| \\
    \verb|sinh(x)|    &   hyperbolic sine of \verb|x| \\
    \verb|tanh(x)|    &   hyperbolic tangent of \verb|x| \\
    \verb|exp(x)|     &   exponential of \verb|x| \\
    \verb|log(x)|     &   natural logarithm of \verb|x| \\
    \verb|log10(x)|   &   base \NUMBER{10} logarithm of \verb|x| \\
    \verb|sqrt(x)|    &   square root of of \verb|x| \\
    \verb|ceil(x)|    &   least integer over \verb|x| \\
    \verb|floor(x)|   &   great integer under \verb|x| \\
    \hline
    \verb|max(x,y)|   &   maximum of \verb|{x,y}| \\
    \verb|min(x,y)|   &   minimum of \verb|{x,y}| \\
    \verb|atan2(x,y)| &   arctan of \verb|y/x| \\
    \verb|pow(x,y)|   &   power $x^{y}$ \\
    \hline
  \end{tabular}
\end{center}


\section{Defining new functions}
%%
A new function can be introduced into the expression
evaluator by defining it as static and then 
passing the evaluator its name and address pointer
by using the two evaluator facilities \METHOD{set\_unary\_fun} 
and \METHOD{set\_binary\_fun}. 
%%
The following example illustrates the mechanism. Let us first
define the two static functions:
%%
\begin{code}
  static double power2(double const a)
  { return a*a ; }
  
  static double add(double const a, double const b)
  { return a+b ; }
\end{code}
%%
Then let us add \verb|power2| and \verb|add| to the current
expression evaluator as follows:
%%
\begin{code}
  ee . set_unary_fun("power2",power2) ;
  ee . set_binary_fun("add",add) ;
\end{code}
%%
These new functions can now be invoked in symbolic expressions 
as the predefined ones:
%%
\begin{code}
  err = ee . parse("power2(add(2,e))") ;
\end{code}
%%
The expression evaluator is capable of handling only unary and 
binary functions, i.e. functions with one or two arguments.

\section{Defining new variables}
%%
New variables can be introduced into the expression evaluator by using
the method \METHOD{set} or the assignement operator.
For example, the following piece of source code defines the new variable 
\verb|abc| and initialize it to the value \verb|1/3| 
%%
\begin{code}
  err = ee . parse("abc = 1/3") ;
  ee . set("abc",1.0/3.0) ;
\end{code}
%% 
The first statements uses the parse method and the assignement operator
\verb|=| of the expression evaluator. The parse method evaluates the 
expression on the right of \verb|=| and then assigns the parsing result 
to the variable on the right. 
%%
If the variable should not exist it would be created and assigned.

The second statement creates -- if needed -- and assigns directly the 
variable. Once created and initilized, the variable can be used in 
the next operations; for example
%%
\begin{code}
  err = ee . parse("zz = abc*sin(3)/(1+abc)") ;
\end{code}
%%
In this case the new variable \verb|zz| is also created.
A variable is a string which always begins with a letter and may be
followed by any sequence of alphanumeric characters, such as numbers, letters or 
the underscore symbols like \verb|_|.  


The \METHOD{exist} return true if its argument is a 
defined variable, as in
%%
\begin{code}
  bool ex1 = ee . exist("abc") ;
  bool ex2 = ee . exist("pippo") ;
\end{code}
%%
In this case \verb|ex1| is set to true and \verb|ex2| to false.
It is possible to get out the value of a variable, 
%%
\begin{code}
  double val1 = ee . get("abc") ;
  double val2 = ee . get("pippo") ;
\end{code}
%%
The value of \verb|val1| is 0.333333 while \verb|val2| is null 
because the variable \verb|pippo| does not exist.

\section{Parsing a file}
%%
The expression evaluator can be used to parse a complete
file. The parsing process proceeds by reading the file 
one line at a time and parsing it. Use the method
%%
\begin{code}
  ee . parse_file("filename", true) ;
\end{code}
%%
The boolean \verb|true| in the second entry asks the 
expression evaluator for proceeding in verbose mode, 
that is for printing out on \verb|cerr| input errors
when detected. If the flag was set to \verb|false|,
reading would proceed silently and errors ignored.

For example, consider the following input file:
%%
\begin{code}
# this is a comment line
gamma = 1.4
# set left state
rin = 1 # density
vin = 0 # velocity
pin = 1 # pressure
ein = rin*vin*vin/2+pin/(gamma-1)

# set right state
rout = 0.125
vout = 0
pout = 0.1
eout = rin*vin*vin/2+pout/(gamma-1)
\end{code}
%%
If a program needs as input parameters \verb|rin|, \verb|vin|,
\verb|ein|, \verb|rout|, \verb|vout|, \verb|eout| the following 
piece of code
%%
\begin{code}
  ee . parse_file("file.data", true) ;
  double rin  = ee . get("rin") ;
  double vin  = ee . get("vin") ;
  double ein  = ee . get("ein") ;
  double rout = ee . get("rout") ;
  double vout = ee . get("vout") ;
  double eout = ee . get("eout") ;
\end{code}
%%
does the work. The advantages of using expression evaluators in reading
input files are multiples:
%%
\begin{itemize}
    \item a free input format is easily usable;
    \item comments can be added everywhere therein;
    \item simple computations may be inserted as part of an
          input file.
\end{itemize}
%%

\end{document}
